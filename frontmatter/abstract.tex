
Ion thermal transport in improved confinement plasmas in the Madison Symmetric Torus (MST), a reversed field pinch experiment, is characterized using a model and compare approach. MST can achieve a improved confinement state by reducing tearing mode fluctuations via pulsed parallel current drive (PPCD). An 1-D model is built to foreword model and predict the evolution of $T_i$ profile during PPCD from plasma parameters and an initial condition, and the result is compared to measurements. The model account for classical transport mechanisms, neutral and charge change effects, and ion flow and compression effects, and its predictions is found to match observations in the core. An \adhoc heating term localized near the reversal surface is needed to fully match observations there. As part of the model construction and analysis, the neutral dynamics in PPCD plasmas are investigated using Monte Carlo modeling via the DEGAS2 code. Detailed characterization of neutral density, temperature, ionization rate and charge exchange loss are obtained as well as their evolution during PPCD. The core neutral are calculated to be several hundred eVs in temperature and this contributes to a hollow charge exchange loss profile and partial recovery of lost energy via ionization of hot neutrals. Evidence of inward particle flux in the core during the early PPCD, likely associated with \ecb pinch are also observed during modeling, and the compressional effects from this \ecb are incorporated into the ion thermal transport model. Finally, the location and extent of the needed \adhoc heating is discussed as well as possible contributing physical mechanisms.

Tearing mode reduction during PPCD are sometimes interrupted by brief and limited excitation's know as m = 0 bursts before returning to quiescence. These burst are observed with sudden radially localized impurity heating near the reversal surface followed by equilibration to previous temperature. Despite strong toroidal localization of the burst fluctuations, the heating does not exhibit toroidal time delay that would be associated with heat transport, pointing to the likelihood of wave or turbulence based heating. Possible anisotropy in heating is investigates and found inconclusive, and the equilibration is found to be faster than what would be expected for the majority ion based on the aforementioned model, possibly indicating that majority heating response is significantly different than the impurity response.
