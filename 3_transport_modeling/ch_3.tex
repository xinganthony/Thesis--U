\chapter{Ion Classical Transport Model Structure and Design}

\section{Why 1-D classical transport?}
Because it's easy. Also $q^2/\epsilon^{3/2} < 0.2$ everywhere, see figure that I have yet to make.

\section{Inputs and assumptions}
Mostly from the diagnostics mentioned in the previous chapter.

\subsection{Ensembling of diagnostic inputs}

\subsection{Geometry}

\subsection{Boundary conditions}

\subsection{Initial conditions}

\subsection{Impurity density and transport}

\subsection{Duel time scale code implementation}

\section{Ion transport terms and their implementation}
Two time scales, based on MSTfit

\subsection{Ion electron equilibration}

\subsection{Classical collisional transport}

\subsection{Neutral "transport": charge exchange loss and electron impact ionization}

\subsection{Flow heating}



