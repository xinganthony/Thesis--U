\subsection{The $D_{\alpha}$ array}\label{sec:dalpha_array}

Neutral particle density profiles are determined by modelling the neutral particle dynamics from particle sources and sinks which are constrained by the measurement of deuterium Balmer-alpha (or $D_{\alpha}$) emissions arising from the $n=3$ to $n=2$ transition. $D_{\alpha}$ emission occurs when a deuterium atom is excited to a higher state, usually through electron impact excitation or charge exchange. The rate of photon generation is
\begin{align}
    \gamma_{D_{alpha}} = \frac{A_{32}}{ A_{32} + A_{31} }\left( n_0 n_e <\sigma v>_{\text{excitation}} + n_0 n_i <\sigma v>_{\text{CX}} \right)
\end{align}
% I changed this equality to a proportionality because there are some factors missing. I'll go over the derivation of the photon emissivity coefficient (PEC) with you to make sure you understand this point. Also, you use a symbol here, $\gamma_D_\alpha$, that never comes up again in this chapter. Perhaps it's better to write this out as the spectral emission coefficient. Again, we can go over the derivation of the PEC.
where $n_0$ is the neutral density, and $<\sigma v>$ are the distribution averaged reaction rates for electron impact excitation and charge exchange into the neutral deuterium $n=3$ excited state and $A_{32}$ and $A_{31}$ are the Einstein coefficients for spontaneous emission. Given $n_e$, $n_i$, and $n_0$ the neutral emission is easy to calculate, but it varies over $3$ orders of magnitude across the plasma diameter with very little emission arising from the hot core. Accurately determining the neutral density in the core requires careful calculations of the neutral dynamics which are discussed in detail in section \ref{sec:DEGAS2}.


\begin{figure}[!htb]
	\centering
	\includegraphics[width = 0.9\linewidth]{./implementation/d_alpha_detector.PNG}
	\label{fig:D_alpha_diagram}
	\caption[$D_{\alpha}$ detectors]{Diagram $D_{\alpha}$ detectors and available viewing chords. The detectors consist of focusing optics, a silicon photo-diode, and a trans-amplifier circuit. In some detectors there is also a pin-hole aperture used to reduce the intensity of incident light on the detector. Five of the viewing cords have matching ports on the other side, but they are not used for this work. (Reproduced with modifications from J. Anderson\cite{Anderson2001})}
\end{figure}

MST uses 13 silicon photo-diode detectors to observe the 656nm $D_\alpha$ line.  The detector array (referred to as the $D_{\alpha}$ array) is located at the $210 \degree$ toroidal boxport. The boxport has 17 available lines of sight, hence we are limited by the avaliable detectors. The detectors consist of filtered silicon photo diodes with a built-in current-to-voltage transimpedance amplifier. Figure \ref{fig:D_alpha_diagram} shows the physical assembly of the detectors.

%But before we leave the topic of $D_{\alpha}$ detectors, a note is needed about it's calibration. 
The detectors are calibrated using a commercially available Tungsten strip-lamp with integrating sphere which has been calibrated for radiance measurements. An optical assembly identical to that on MST is used to couple light from the integrating sphere to the detector. Since the optical geometry is reproduced, the geometrical effects quantified by the {\'e}tendu $A\Delta\Omega$ are held to be constant between calibration and machine. However, the transfer function is much wider than the $H_{\alpha}$ emission line, and thus needs to be taken into account. Namely the voltage measured in calibration is
\begin{align}
    V_{\text{cal}} &= c A \Delta\Omega \int^{\infty}_{0}f(\lambda)\ d\lambda R\\
    &\equiv c'R
\end{align}
where c and c' are calibration factors, $f(\lambda)$ is the transfer function of filter, and R is the radiance of the integrating sphere. With c' thus defined, it can be easily calculated from measured pairs of output voltage vs luminance slope from,
\begin{align}
    c' &= \frac{V_{\text{cal}}}{R}\\
    &= \frac{\Delta V}{\Delta L}\left(\frac{L}{R}\right)_{\lambda}
\end{align}
where $\frac{\Delta V}{\Delta L}$ is the measured voltage vs. luminance slope (see Figure \ref{fig:v_vs_l}), and $(\frac{L}{R})_{\lambda}$ is the luminance to spectral radiance conversion factor provided by NIST-certified calibration of the light source. To apply to measurements, however, we have to further consider that the voltage due to measurement of plasma emission is
\begin{align}
    V_{\text{meas}} &= c A\Delta\Omega\int^{\infty}_0\int_{C} f(\lambda)\epsilon(\lambda,l)\ dld\lambda
\end{align}
where C is the view path, l is the distance along the line-of-sight, and $\epsilon$ is the spectral emissivity. Since the plasma emission is a discrete emission line and the calibrated light source is a broad-spectrum source, we cannot equate the integration over spectral wavelength between the calibration and plasma measurements. Thus, an additional factor is needed. Approximating the spectral emissivity as a delta function, {\em i.e.} $\epsilon\approx I_{D_{\alpha}}(l)\delta(\lambda - \lambda_{D_{\alpha}})$ where $I_{D_\alpha}$ is the emissivity, the measured voltage becomes
\begin{align}
    V_{\text{meas}} &= c A\Delta\Omega f(\lambda_{D_{\alpha}}) \int_{C}I_{D_{\alpha}}(l)\ dl \nonumber\\
    &= c' \frac{f(\lambda_{D_{\alpha}})}{\int_0^{\infty}f(\lambda)d\lambda}\int_C I_{\dal}(l)\ dl \\
    &\equiv c' c_{\text{trans}}\int_C I_{\dal}(l)\ dl 
\end{align}
where in addition of the calibration factor $c'$ defined previously, the transfer function normalization factor $c_{\text{trans}} \equiv \frac{f(\lambda_{D_{\alpha}})}{\int_0^{\infty}f(\lambda)d\lambda}$ is also need for the proper calculation of the line emission intensity. This latter factor $c_{\text{trans}}$ was precisely measured using a high-resolution Ocean Optics HR2000+ spectrometer as the transfer function of the filters showed noticeable variations\cite{Eilerman}, but it is not regularly re-calibrated as it would require disassembly of the detectors. The radiance calibration of $c'$ using the integrating sphere is performed once a year during the data taking period for this work. More details of the hardware and calibration process can be found in Anderson's Ph.D. thesis\cite{Anderson2001} (Chapter 2) and Eilerman's Ph.D. thesis\cite{Eilerman} (Appendix A).