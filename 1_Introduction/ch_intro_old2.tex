\begin{refsection}

\chapter{Introduction}

Fusion energy is the way of the future, and to make the future a reality, the problem of confinement have to be solved. 
A wide variety of alternative confinement
strategies have been attempted, broadly separated into magnetic confinement and
inertial confinement. Where as the inertial confinement focuses on short,
repeated pulses of fusion reaction at very high densities driven by enormous
lasers, magnetic confinement focus on using the magnetic field to restrict the
movement of plasma particles long enough to produce net fusion power. In order to produced net energy from fusion, the ion heat energy have to be sufficiently confined that it could be heating to and maintained at the necessary high temperature without a need for excessive external heating. This requirement often expressed as the Lawson Triple Product:
\begin{align}
    nT\tau_{E} &\geq 10^{24} [eV m^{-3}s] &\text{  for T $\approx$ 15KeV D-T fusion.}
\end{align}
where n is the density, T is the temperature and $\tau_{E}$ is the energy confinement time of the D-T ions. Of the three values, the confinement time is the one that we have the least direct control over, and in many ways, the story of fusion research is the story of trying to figure out why $\tau_{E}$ isn't as high as physicists hoped. 

To that end, physicists works hard find and identify the processes by which heat is lost from the plasma, and these are called heat transport mechanisms. Transport can be thought of as a complementary concept to confinement.Where as confinement is often used to refer to the broad quality of 'keeping the energy in the plasma', and characterized by empirically measured $\tau_{E}$ values; and transport refers to the specific mechanisms, known or unknown, by which energy or particle leaves the plasma volume. Transport mechanisms in magnetic fusion is broadly separated into classical, neoclassical, and turbulent transport. There is, of course, still an gap in understanding, between the loss mechanisms that we know and can calculate, and the overall confinement we observe. This gap is described as anomalous transport; often there is anomalous heat loss, but occasionally also anomalous heating. The topic of plasma confinement and transport is vast and will likely take several careers to fully study. This work seeks to quantify the various ion transport mechanisms active in one particular type of operation in one particular type of magnetically confined fusion plasma.

\section{Ion confinement in magnetically confined fusion plasmas}

In an ideal world, plasmas would be collisionless, fields would be straight and infinite, there would not be drifts or gradients, and then, there would be no transport. Alas, that is not what we have to work with. The transport mechanisms are usually divided based on which non-ideal factors are introduced back into consideration. With collisions, gradients, and an infinitely repeating cylindrical magnetic geometry, we get classical transport; add toroidicity and the banana orbits that results, we get neo-classical transport; add magnetic wave and turbulence, we get turbulent transport. In addition to these, there is also transport and transport-like processes that don't neatly fall into any category, and then there is the anomalous transport, from processes that we either don't understand, or are not sure of.

\subsection{Classical transport}

Classical transport arises from collisions, and thus the crucial quantity is the collision frequency. To derive it, on has to first consider the fact that sub-atomic particles do not collide like billiard balls, but instead through Columb (electro-static) interactions

\subsection{Neo-classical transport}

\subsection{Turbulence and anomalous transport}

\subsection{Other transport processes}

\section{The reversed field pinch and ion confinement challenges}

RFP ion confinement in standard plasmas, sawtooth, anomalous heating and such.

\section{Improving confinement with inductive current profile control}

PPCD, it's effects, it's characteristics, bursts in PPCD.

\section{What this thesis is about: 1-D classical transport modeling}


As for the structure, I start out with a discussion...
Make no mistake, the interesting parts of this thesis are in chapter \ref{ch:results} and chapter \ref{ch:m0}. However the preceding chapter is an adventure in what went into getting those results and why one should put any weight on them.





\end{refsection}