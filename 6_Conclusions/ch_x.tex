\begin{refsection}


\chapter{Conclusions and remarks}

\section{Summary of results}

A modeling approach is used to characterize ion thermal transport in improved confinement, reduced tearing (PPCD) plasmas in MST. The forward model is constructed in 1-D approximation and includes classical transport physics, charge exchange and particle flow. The model uses a range of diagnostic inputs to provide the plasma profiles (of $T_e$, $n_e$,$n_D$ as well as reconstructed magnetic) to predict $T_i$ evolution from an initial condition. This model is compared to measurement and found to be sufficient in predicting the core temperature evolution, but needed an \adhoc heating source localized in the gradient region to match observations. Additional thermal transport is not required to match measurements. The model includes the calculation of electron-ion equilibration, classical conduction, charge exchange loss, and flow related heating (including compression and convective heat flow). The model is applied to an ensemble of selected 500kA PPCD shots in MST, and it's temperature predictions are compared to measurements of $T_{C^{6+}}$ as a proxy of $T_i$.

The neutral dynamics is found to be important in constructing and analyzing the model. A physics based Monte Carlo simulation is used to calculate the neutral density and charge exchange loss, especially in the core. The simulations show that the core neutrals are 'warm' ($\approx 0.6-0.8 T_i$) and the charge exchange loss profile is hollow. Further, a portion of the energy lost to the neutrals is recovered via subsequent re-ionization. This simulation also provides the ionization source rate profile, which when combined with density measurements show that an inward particle flux in the core is necessary to account for the core density rise early in PPCD. Comparison is made with estimates of \ecb flux in PPCD and found to be a match. The compression effect of this flux is incorporated into the model, as well as the effect of the edge particle outflow. 

%Mark's email comments
%Make sure to list all of the unique things that you can point to which you did in your conclusion chapter. Keep in mind that you have done the most thorough analysis of ion heat transport in PPCD plasmas to date. You’ve done the most thorough analysis of neutral density profile evolution during PPCD which informs not only your ion power balance work, but has also been important in predicting charge state distributions of highly charge ions (like Aluminum) for predicting x-ray emission spectra (Patrick & Lisa’s work for the ME-SXR model and now the LLNL x-ray calorimeter). You’ve shown that any heating beyond classical mechanisms that you’ve called out in your model are edge localized. Your analysis of the m=0 burst activity also suggests that burst heating of the impurities may follow the same type of collisional equilibration with the deuterium that describes the ion heating due to magnetic reconnection present during edge-localized tearing mode excitation in standard plasmas.
%---

An \adhoc heating term is added to the model to match observations. The needed \adhoc term is located near the edge in the gradient region

The model's ion confinement time is found to be , and the largest loss mechanisms are found to be associated with particle loss out of the plasma, with charge exchange a near second. 

The phenomenon of impurity heating associated with m = 0 bursts in PPCD is investigated and found to have interesting properties such as edge locality, (near) simultaneous heating regardless of toroidal or radial location (except where there is no heating), and a faster than expected equilibration time. However, there are still many unclear aspects that surrounds this phenomenon that needs further investigation to clear up. 


This works presents the most thorough accounting for ion thermal transport in PPCD plasmas, and 



\section{Future work}
A more detailed assessment of turbulent heating mechanisms available 
Rutherford scattering measurements of D+ temperature, especially during the 
Direct measurements of neutral temperature. 


\printbibliography
\end{refsection}