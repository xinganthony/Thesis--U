\begin{refsection}


\chapter{Conclusions and remarks}

\section{Summary of results}

A modeling approach is used to characterize ion thermal transport in improved confinement, reduced tearing (PPCD) plasmas in MST. The forward model is constructed in 1-D approximation and includes classical transport physics, charge exchange and particle flow. The model uses a range of diagnostic inputs to provide the plasma profiles (of $T_e$, $n_e$,$n_D$ as well as reconstructed magnetic) to predict $T_i$ evolution from an initial condition. This model is compared to measurement and found to be sufficient in predicting the core temperature evolution, but needed an \adhoc heating source localized in the gradient region to match observations. Additional thermal transport is not required to match measurements. The model includes the calculation of electron-ion equilibration, classical conduction, charge exchange loss, and flow related heating (including compression and convective heat flow). The model is applied to an ensemble of selected 500kA PPCD shots in MST, and it's temperature predictions are compared to measurements of $T_{C^{6+}}$ as a proxy of $T_i$.

The neutral dynamics is found to be important in constructing and analyzing the model. A physics based Monte Carlo simulation is used to calculate the neutral density and charge exchange loss, especially in the core. The simulations show that the core neutrals are 'warm' ($\approx 0.6-0.8 T_i$) and the charge exchange loss profile is hollow. Further, a portion of the energy lost to the neutrals is recovered via subsequent re-ionization. As a side benefit of this characterization of the neutrals in PPCD, the profiles have been used in recent works regarding soft x-ray modeling via the calculation of charge state distribution of high charge impurities. This simulation also provides the ionization source rate profile, which when combined with density measurements show that an inward particle flux in the core is necessary to account for the core density rise early in PPCD. Comparison is made with estimates of \ecb flux in PPCD and found to be a match. The compression effect of this flux is incorporated into the model, as well as the effect of the edge particle outflow. 

%Mark's email comments
%Make sure to list all of the unique things that you can point to which you did in your conclusion chapter. Keep in mind that you have done the most thorough analysis of ion heat transport in PPCD plasmas to date. You’ve done the most thorough analysis of neutral density profile evolution during PPCD which informs not only your ion power balance work, but has also been important in predicting charge state distributions of highly charge ions (like Aluminum) for predicting x-ray emission spectra (Patrick & Lisa’s work for the ME-SXR model and now the LLNL x-ray calorimeter). You’ve shown that any heating beyond classical mechanisms that you’ve called out in your model are edge localized. Your analysis of the m=0 burst activity also suggests that burst heating of the impurities may follow the same type of collisional equilibration with the deuterium that describes the ion heating due to magnetic reconnection present during edge-localized tearing mode excitation in standard plasmas.
%---

An \adhoc heating term is added to the model to match observations. The needed \adhoc term is located near the edge in the gradient region peaking at the reversal surface. The volume integrated \adhoc power from this is about 144 kW, which may be from from several sources of energy available in the region, such as reconnection heating associated with residual tearing mode activity in core resonant modes, and non collisional transfer of energy from electrons due to turbulent mechanisms such as ion Landau damping of driftwaves, or viscous stress heating via zonal flow. Alternatively, it is possible that this \adhoc term is the result of a combination of uniform anomalous heating and a proscribed anomalous diffusion, though it is a less satisfying solution as it requires two mechanisms matching well enough to masquerade as if one. The model's ion confinement time is found to be ~12ms during most of the PPCD period, excepting the onset of improved confinement period. This is close to previously measured confinement time of electrons. The largest loss mechanisms are found to be associated with particle loss out of the plasma, with charge exchange a near second. 

The phenomenon of impurity heating associated with m = 0 bursts in PPCD is investigated and found to have interesting properties such as edge locality, (near) simultaneous heating regardless of toroidal or radial location (except where there is no heating), and a faster than expected equilibration time. However, there are still many unclear aspects that surrounds this phenomenon that needs further investigation to fully understand, such as the degree and direction of anisotropy of the heating and the relation between impurity and majority heating and confinement. 

\section{Future work}
There are many directions in which further investigations would be beneficial. First,the neutral modeling makes the interesting prediction of high temperature neutrals in the core and it would be of interest if this can be directly measured and verified. 

Secondly, The question of what is responsible for the \adhoc heating needed in the model is an important if difficult line of enquiry for the purpose of fully understanding ion thermal transport. A number of mechanisms are available for reversal surface localized heating, as well as the possibility of a combination of transport and diffuse heating. In order to investigate this topic a detailed characterization of plasma turbulent is likely needed. The recent measurement of driftwaves and zonal flow are important first steps but not yet sufficient to answer the question.

Thirdly, it would be interesting to apply this model to different but similar plasma regimes, such as crash or NBI heated PPCDs, or low current PPCDs in order to investigate the robustness of the model as well as differences in plasma behavior.

Finally, a number of open questions exist surrounding the impurity heating from m = 0 bursts. Direct measurement of $D^+$ temperature would be particularly useful, in addressing the fast equilibration time of the heated impurities as described in section \ref{sec:m0_v_model}. The persisting hardware difficulties of the Rutherford scattering diagnostic prevented the incorporation of such measurements. The confusing results regarding the anisotropy in m = 0 heating needs further observation to be resolved, by alternative diagnostic techniques or a through characterization of the evolution of the carbon emission shell changes during an m = 0  burst. Further, more detailed modeling would be needed to asses the expected (majority) response to the heating, after the change in density and electron temperature are accounted more. To do this properly, the model would likely need to be updated such that neutral simulations can be done more frequently to better follow the rapidly changing plasma conditions relating to bursts. 





\printbibliography
\end{refsection}