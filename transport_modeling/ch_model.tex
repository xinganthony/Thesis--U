\begin{refsection}


\chapter{Modeling ion transport in the RFP}\label{ch:physics}

% I think you should mention that prior work on understanding on ion heating and heat transport has concentrated on understanding the mechanisms of rapid magnetic reconnection leading to rapid anomalous ion heating while also leading to rapid transport of electron heat. Cite:    % * Chapman PPCF 52, 124048 (2010)
% * Gangadhara S, Craig D, Ennis D A, Den Hartog D J, Fiksel G
%     and Prager S C 2007 Phys. Rev. Lett. 98 075001
% * Gangadhara S, Craig D, Ennis D A, Den Hartog D J, Fiksel G 
%     and Prager S C 2008 Phys. Plasmas 15 056121
% What's new here is that you are considering the ion heat transport under quiescent conditions in the best performing MST plasmas and that even in the absence of strong, rapid reconnection events there are heating mechanisms still present, particularly in the edge of the plasma. Then cite   

A full understanding of ion transport is a broad topic and difficult to tackle. Past efforts to understand transport in MST has concentrated on the sawtooth crash and it's effects on transport. The magnetic reconnection of the sawtooth events leads to rapid and anomalous ion heating as well as rapid heat loss from electrons\cite{Chapman2010}
%WORK IN PROGRESS

To make the problem more tractable, it was decided to focused on the improved confinement regime achieved through PPCD. Unlike improved confinement through Quasi-Helical State (QSH), PPCD plasmas are highly axisymmetric and have reduced stochasticity, making it ideal for a simplified 1-D model. Previous work have established preliminary 0-D and 0.5-D power balance for the ions\cite{??} which are limited in usefulness as it cannot distinguish the likely difference that exists between the core and the edge/gradient region. A 1-D model is the logical next step in investigation ion transport and anomalous heating. This chapter aims to examine the ion thermal transport physics being included in the model and some of those that are not. Starting with the relatively simple to calculate, I will briefly review thermal equlibration between electron and ions, as well as classical collisional thermal diffusion. Then I will discuss the effects of neutral particles and the particle flow on transport. The proper treatment of neutrals are important due to a combination of the fact that the other active terms are small in PPCD. Further, charge exchange neutrals are not confined by magnetic fields, giving then outside influences on a per reaction basis, as well as unintuitive behaviors. This is followed by a discussion of neoclassical and stochastic effects on heating transport and why they are not modeled by the code. At the same time, I aim to separate in the reader's mind, their effects on particle transport vs on thermal transport, and how they are ultimately treated in the final comparison with data. 

Note that this chapter has a large number of viable being introduced, and it is not always practical to note their meaning with every equation. Please refer to appendix \ref{app:term} where one would find notes on notation and terminology. 

\section{Modeled ion transport mechanisms}
This model is an attempt to catalog and characterize heating terms that are known to be active and isolate and describe the extent of what we do not know (ie. anomalous heating), and from there, investigate mechanisms that would produce the anomalous heating. In this section, I explore the known heat transport mechanisms being modeled and their general physics basis and characteristics. Some of these mechanisms also couples with particle density change, which may result in the phenomenon where the power is positive (ie, ion thermal energy increasing) but the ion temperature is decreasing. This is discussed in more detail in appendix \ref{app:term}

\subsection{Ion electron equilibration}

In a plasma where the temperature is different between two species, Coulomb collisions causes a transfer of thermal energy from the hotter to the colder. This is thermal equlibration. In PPCD plasmas, the electron confinement is significantly improved compared to standard RFP, and $T_e$ rises significantly both compared to the pre-PPCD period, to levels significantly above $T_i$. Thus, equlibration is a heating term for the ions. This is a reflection of the fact that MST is (primarily) Ohmic heated device, which is nearly entirely an electron heating mechanism. Particularly in PPCD plasmas where anomalous heating mechanisms available to ions are reduced, the thermal equalization becomes important. The equalibration power can be calculated as:
\begin{align}
    P_{ei} &= \frac{3}{2}n_i\nu^{i/e}(T_e - T_i)\\
\end{align}
where $\nu^{i/e}$ is the thermal equalibration frequency, defined:
\begin{align}
    \nu^{i/e} = 5.69\times10^{-27}\frac{(m_e m_i)^{1/2}(Z_e Z_i)^{2} n_e ln \Lambda}{(m_iT_e + m_eT_i)^{3/2}}
\end{align}
It is important to note that $\nu^{i/e}$ is not the Lorentz collision frequency but is related to the energy relaxation frequency. Due to the mass discrepancy between ions and electrons, energy transfer between the two is small per collision. Thus, the calculated equlibration heating from the even the lower temperature edge region is relatively small even though the collision frequency, in the normal sense, is relatively fast. This fact is behind the fact that in PPCD $T_e$ is 'allowed' to rise far above $T_i$. A typical profile is shown below:

\begin{figure}[!htb]
	\centering
	\includegraphics[width = 0.75\linewidth]{./transport_modeling/p_eq.png}
    \label{fig:P_eq}
    \caption[Example equlibration power profile.]{An example of equlibration profile in PPCD plasma conditions. The plot also show the $T_e$ and $T_i$ that forms the basis of this calculation.}
\end{figure}%

\begin{figure}[!htb]
	\centering
	\includegraphics[width = 0.75\linewidth]{./transport_modeling/tau_eq.png}
    \label{fig:P_eq}
    \caption[Electron ion thermal equlibration time]{An example of equlibration profile in PPCD plasma conditions. The plot also show the $T_e$ and $T_i$ that forms the basis of this calculation.}
\end{figure}%

This heating term would increase slightly as the PPCD period progress, due to the increase in $T_e$ and the relative constancy of $T_i$. However, the increase in heating is smaller than what might be expected at first sight as the increase in temperature decreases the collisionality. This may indicate that Ohmic heating is not efficient for ion heating. 

\subsection{Classical thermal conduction}\label{sec:thermal_cond}

Classical thermal conduction refers to the heat diffusion due to Coulomb collisions. Note that this is not the same as the particle transport and the thermal energy associated with that movement. In an cartoonish approximation, thermal diffusion is caused by a hotter ion colliding with a colder ion on an adjacent field line during their gyro-orbits. Energy, or possibly position is exchanged, however, the center-of-mass velocity is conserved and thus no net particle flow results. The thermal conduction is thus dependent on the temperature gradient and can be calculated as\cite{Braginskii1965}:
\begin{align}
    P_{cond} &= \nabla\cdot\vec{q}_{\text{cond}} \\
    &= \frac{1}{\rho}\frac{\partial}{\partial\rho}(\rho\kappa_{\perp}\nabla T_{i})\\
    \kappa_{\perp} &= \frac{2n_ikT_i\nu_i}{m_i\omega_{ci}^2}
\end{align}

In comparison with the other terms in play, the thermal conduction term is small and can largely be ignored. It is, however, simple to calculate and still included in the model. A typical $P_{\text{cond}}$ profile is presented below. As one would expect, the term is most active in the edge where the gradient is steep and collisionality is high.

\begin{figure}[!htb]
	\centering
	\includegraphics[width = 0.75\linewidth]{./transport_modeling/p_cond.png}
    \label{fig:P_eq}
    \caption[Thermal conduction profile]{An example of thermal conduction profile. Note the axis, this term is significantly below the equlibration term presented above, and most of the terms that I'll present later.}
    \label{fig:p_cond}
\end{figure}%

\subsection{Charge exchange and neutral transport}\label{sec:neutral_physics}

MST's thermal transport is significantly affected by the neutral population through charge exchange. Charge exchange is the process where a neutral particle, typically atomic deuterium for MST, 'gives' it's electron to an ion due to a collision. This causes the hot ion to become neutral and thus un-confined. Although the neutral fraction of the plasma is very low, charge exchange can have an out-sized effect due to the neutral essential 'by-passing' the entire magnetic confinement concept. This is especially significant in improved confinement (PPCD) plasmas where other mechanisms are reduced. The charge exchange loss can be calculated as:
\begin{align}
    P_{CX} = n_0 n_i <\sigma v>_\text{CX} (T_i - T_0)
\end{align}
where $T_i$ is the ion temperature, $T_{0}$ is the neutral temperature.

In previous estimates, the neutral fluid is assumed to be cold, therefore, exchange collisions tends to equlibrate ions with this "cold" neutral fluid. However,the mean free path of a "cold" neutral is
very short (<1cm), but edge neutrals may undergo subsequent charge exchange collisions resulting in higher temperature neutrals that can penetrate into the plasma. Consequently, neutrals in the core are mostly generated near the mid-radius and have a temperature comparable to mid-radius ions. At the same time, a charge exchange neutral created from thermal neutral in the core, if traveling through core-like conditions, has a mean free path
shorter than the minor radius, implying a fraction of such neutrals would undergo secondary ionization or charge exchange reactions. 

\begin{figure}[!htb]
	\centering
	\includegraphics[width = 0.75\linewidth]{./transport_modeling/neutral_n.png}
    \label{fig:n_n_0}
    \caption[Typical neutral density profile]{A typical neutral density, as you can see, there is nearly 3 orders of magnitude drop between edge and core densities. This makes the determination of the core density difficult without modeling efforts. The analysis that produced this density profile are explained in more detail in section \ref{sec:neutral_dynamics}}
\end{figure}%

As far as terminology and conceptualization is concerned, the charge exchange 'transport' is presented as two terms in the modeling. The first is the charge exchange term itself $P_{CX}$, in which when an charge exchange reaction occurs, the difference in energy is considered lost from the ion fluid immediately. This lost energy is not entirely lost, but instead should be thought of as part of the neutral fluid's thermal energy. This thermal energy both reduces the $P_{CX}$ by reducing the temperature difference between the ion and the neutral fluid, but also may undergo subsequent electron impact ionization, by which the associate thermal energy will come back to the ion fluid. This is marked as a separate term noted as $P_{e-imp}$. It is important to note that while the electron impact brings the thermal energy 'back' to the ion fluid, it does not (typically) cause heating as the process is also a particle source term that (typically) is adding a cooler ion to the fluid. The consequences of mechanisms that contributes both particle and heat are discussed more directly in Appendix \ref{app_sec:power_terms}.

From a diagnostic point of view, the neutral density is not a straight forward quantity to measure. The neutral density drops precipitously towards the core of the plasma where the temperature is high. Measurements of neutral emissions, in MST's case the deuterium Balmer-alpha line emission, only reflects the edge neutral density as that's where the emission is coming from. The emission from the core is so low such that it is buried in the noise and fluctuations of the edge emission. Therefore, physics-based equilibrium modeling is needed to construct self-consistent core neutral profiles that produce the observed $D_{\alpha}$ emissions. Temperature would be even more difficult to measure as it's not accessible with typical spectroscopic methods. Recently on the HIT-SI3 experiment, researchers have been able to use a two-photon Laser Induced Florescence diagnostic to directly measure the density of temperature of deuterium neutrals\cite{Elliott2016}. However, such a diagnostic is not available for MST and would likely be unable to measure the low neutral density expected in MST's core. 

Instead, Monte Carlo modeling are used to solve the issues detailed above. Monte Carlo modeling uses representative particle which are drawn from a distribution. Their location and velocity is tracked, and relevant physical interactions and reaction are simulated with appropriate physics based probability distributions. The 'fate' of any given particle is fairly random is not particularly illustrative, but the simulation uses a large quantity of test particles for the simulation such that an adequate number and distribution is reached in the resulting quantities of interest. For this work, the particular quantity of interest is not only the neutral density and temperature, but also the two power terms described above( $P_{\text{CX}}$ and $P_{\text{e-imp}}$) which are tallied through the simulation directly. The simulation code used to achieved this is DEGAS2 developed at PPPL\cite{Stotler}, and the details of it's implementation as part of the model are detailed in section \ref{sec:DEGAS2}. The result from the neutral simulation regarding the charge exchange loss and electron impact ionization is complex and will be presented more fully in section \ref{sec:neutral_results} after the implementation of DEGAS2 have been described. 

\subsection{Particle flow, compression, and decompression}\label{sec:flow_effects}

When an ion moves, it carries its thermal energy with it. In this section I will speak generally regarding flow, but in the modeling work only the radial flow are considered as the symmetry assumptions implies no meaningful effects from toroidal and poloidal flow. The particle continuity equation can be written as:
\begin{align}
    \partialt n + S_{tot} = -\nabla\cdot(n\vec{V})
\end{align}
where $S_{tot}$ is the total source term. Extending the continuity equation to the thermal energy carried by the particles, and ignore the source term by focusing on the effect of the particle flow, we get:
\begin{align}\label{eqn:thermal_continuity}
    \partialt nT = -\nabla\cdot(nT\vec{V})
\end{align}
In an compressible fluid, however, there is the potential of compression work by external sources, so the equation becomes:
\begin{align}
    \frac{3}{2}\partialt nT &= -\frac{3}{2}\nabla\cdot(nT\vec{V}) - nT\nabla\cdot\vec{V}\\
    &= P_{\text{flow}}\nonumber
\end{align}
where the right side term is the \emph{compressional work} term, and the left side term is the \emph{flow conservation} term. For convenience of reference in the rest of this work, we define $P_{\text{comp work}} \equiv - nT\nabla\cdot\vec{V} $, and $P_{\text{flow cons}} \equiv -\frac{3}{2}\nabla\cdot(nT\vec{V})$.


\subsubsection{A discussion on which velocity is proper in calculation thermodynamic compressional work}

The compressional work term needs a more detailed examination from the fact that not all net decompression would invoke the work term. This makes it important the 'correct' velocity is used to calculate the work term. To begin, consider the difference between adiabatic expansion and Joules free expansion. In adiabatic expansion, an gas in an fully insulated chamber expanded against a 'piston' and does work on it. The net power on the gas is thus $P_{\text{work}} = - PdV$ where $P$ is pressure and $V$ is volume. Joule free expansion, is where there is two adjacent chambers, one of which is filled with a gas with finite temperature and pressure, and the other is in perfect vacuum. The two chambers are perfectly insulated. At a $t=0$ the wall between the two chambers are broken and gas from the first chamber is allowed to flow to the second until the pressure equlibrates. In this process, the expansion does no work, and not net energy is lost, it merely moved to be spread over a larger space. 

To examine the implications on plasma, we can start with a thought experiment. Imagine a  slab plasma adjacent to a perfect vacuum confined by a static magnetic field at equilibrium. The plasma have finite temperature with no transport mechanisms active except for what is about to be specified. The plasma is not perfectly confined, and a small flux of particles are being lost to the vacuum. In this situation, the only energy lost is that carried by the particle flux, without a work term. As a result, the temperature (energy per particle) of this plasma would not change. If the $- nT\nabla\cdot\vec{V}$ work term was to be applied in this situation, then the temperature of the remaining plasma would cool, which would conflict with the understanding of microscopic mechanics of plasma. This situation would be analogous to the Joules free expansion mentioned above, with the 'wall' in the joules free expansion case replaced with an imperfectly confining magnetic field. By a similar analogy, plasma compression via the movement of a perfectly confining magnetic flux surface can be compared to a compressing piston. It's useful to note that there is no free compression. If compression occurs, then \emph{something} has done work, and it's a matter of finding what. This is also why free expansion is considered an irreversible process, as there is no equivalent reversing process.

In real PPCD plasmas, however, these two mechanisms are not so easily separated. During the early phase of PPCD, the reversal surface is observed to move inwards, as does the $n_e$ gradient region. As will be discussed in section \ref{sec:eb_pinch}, there is an inward pinch likely related to an $\vec{E}\times\vec{B}$ drift. However, in the outer half of the plasma the net particle flow is outwards across flux surfaces due to particle transport mechanisms outside of the consideration of this thesis. This, in the thermodynamic analogy, would be akin to a leaky piston moving inwards, but the net particle velocity being outwards. The velocity use to calculate $P_{\text{flow cons}}$ should be the net velocity, as the the ions will carry their energy with them whatever the 'reason' of their movement. However, the velocity used in the work term should be the velocity associated with the mechanism by which the work is done. In this case, the \ecb drift and the movement of the flux surface associated. The implementation and consequences of this, as well as the determination of the \ecb flux and it's association with compression is discussed in much more detail in section \ref{sec:eb_pinch}.

\section{Unmodeled terms and mechanisms}

\subsection{Neoclassical effects}
\begin{figure}[!htb]
	\centering
	\includegraphics[width = 0.75\linewidth]{./transport_modeling/banana_width.png}
    \caption[Banana width and ion trapped fraction in PPCD]{Banana width and ion trapped fraction in high current PPCD. This also qualitatively applies to nearly all RFP plasmas. The primary 'cause' of this is the low $B_t$ that RFPs use to confine the plasma.}
    \label{fig:banana_width}
\end{figure}%
Neoclassical effects are, in approximation, those that arises due to the toriodal nature of the magnetic confinement devices. In short, due to the nature of Ampere's law, $B_t$ on the inboard side is higher than the low field side, and drops as $~1/R$. This fact is important enough in the Tokamak world that the inboard side is typically referred to as the high field side, and the outboard side being the low field side. As a given field-line travels poloidally around the plasma volume, particles see varying $|\vec{B}|$ and effectively experiences a magnetic mirror. The portion of particle with in sufficient $B_{\parallel}$ to complete the journey around poloidally are referred to as trapped. They are reflected by the mirror effect and 'attempts' to retrace it's steps. But $\nabla B$ and curvature drifts give the poloidal path of the particle a certain width, giving it a crescent shape. This is called the banana orbit, and the width the banana width. For the Tokamak geometry, the existence of the banana orbit enhances the collisional transport experienced at low collisionality regimes since a collision may move a particle a banana width from it's previous orbit average location instead of a gyro-radius. This effect goes away as the collisionality increase as the particles are no longer able to complete banana orbits before collisions alters their path. Further, the collisional transport is typically not \emph{directly} effected by neoclassical considerations. While the banana width is larger than the gyro-radius in Tokamak geometry, that is not typically the case for RFPs including the MST. The average banana width only exceeds the gyro-radius near the core, where the trapped fraction is small. In the low collisionality Pfirsch-Schluter regime, the neoclassical diffusion coefficient is to a first approximation $D_{P-S} \approx \frac{q^2}{\epsilon^{3/2}}D_{\text{classical}}$

\begin{figure}[!htb]
	\centering
	\includegraphics[width = 0.75\linewidth]{./transport_modeling/neo_class_ratio.png}
    \caption[$\frac{q^2}{\epsilon^{3/2}}$ in PPCD]{A plot of $\frac{q^2}{\epsilon^{3/2}}$. It is under 0.2 for the entire plasma, thus the neoclassical transport is below the classical diffusive transport for entire volume, and could be safely ignored.}
\end{figure}%

The trapped particle populating will also affect the plasma stability, turbulence, and thus effect the transport properties indirectly. For example, the neoclassical correction to the Spitzer resistivity is important for the electron thermal transport, and trapped-electron mode turbulence is also known and measured for PPCD plasmas. These are not with in the scope of this thesis. However, they are not so much ignored but Incorporated indirectly. The electron temperature is measured and used as an input to the model, so it effectively accounts all electron physics at or slower than the cadence of measurement. The effect of any induced particle transport is indirectly incorporated through the particle continuity equation, and the determination of net particle flux, discussed in section \ref{sec:eb_pinch}

\subsection{Stochastic effects}\label{sec:stochastic_effects}
No discussion of transport in the RFP is complete without a discussion of the effects of stochasticity. However, the full effects of stochasticity is not known to an analytically extent. In their well known paper, Rechester and Rosenbluth\cite{Rechester1978} detailed a way to compute thermal diffusion due to stochasticity. However, previous work on electron thermal transport in MST finds that it poorly describes the observations made \cite{Reusch2011}. To more accurately understand the effect of stochasticity, thesis-es worth of complex simulation modeling work needs to be done. Further, stochasticity is not static, but inextricably linked with fluctuations and turbulence. In the case of ions, stochasticity is proposed as a heating mechanism associated with sawtooth events\cite{Fiksel2009}. Being that stochasticity is both reduced in the PPCD plasma, and that it is likely associate with the anomalous heating that I aim to isolate and investigate, I take the approach of categorizing the stochastic effects on heat transport as a part of the anomalous heating. An quantification and localization of the anomalous heating is instructive in the discussion and investigation into stochastic mechanisms, and this will be discussed in more detail in section \ref{sec:anomalous_heating}.

It is important to note that similar to neoclassical effects on particle transport, the stochastic effects on \emph{particle transport} is indirectly included via the continuity equation, as the density observations are treated as an input parameter. Whereas the stochastic effects on \emph{thermal transport} is treated as anomalous.

\section{Summary of ion thermal transport in the RFP}

In this chapter I have laid out the ion thermal transport physics incorporated into the model. Starting with the simple, classical thermal equlibration between electron and ions, and the classical collisional thermal diffusion, both of which are well known and easy to calculate. The transport effects of the neutral fluid and particle flows are more complex, and as you will see, forms a significant part of the work of this thesis work. The proper treatment of the neutrals and especially the fact that they have a temperature is important to quantifying their effects, and in the result section I will demonstrate that the $P_{CX}$ can exhibit behaviors that are not intuitive, especially within the $Q = -n\chi\nabla T$ framework where one seeks to quantify transport with a calculated $\chi$ value. 

A numerical model is built to predict 1-D ion temperature profile by calculating these terms. The model implementation will be discussed in detail in the next chapter, along with the diagnostics used. However, at it's core, the model numerically integrates the following differential equation,
\begin{align}
    \partialt E_\text{thermal} = P_\text{e-i} + P_\text{cond} + P_\text{CX} + P_\text{flow} + P_\textit{ad hoc}
\end{align}
In the simplest form, the model is a numerical integration of this equation in time, arriving at temperature predictions. The various terms being integrated at calculated at each step with the appropriate diagnostic input and equilibrium reconstruction. The end goal being a comparison of the model prediction with measurement of $T_i$ provided via CHERS. Having explored the physics being modeling in this chapter, I will go into details about the implementation in the next.  


\printbibliography%[title={Section bibliography}]
\end{refsection}

