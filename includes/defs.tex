% defs.tex -- wbepi environment for chapter epigraphs and other useful defs.
%
% Wisconsin dissertation template
% Copyright (c) 2008 William C. Benton.  All rights reserved.
%
% This program can redistributed and/or modified under the terms
% of the LaTeX Project Public License Distributed from CTAN
% archives in directory macros/latex/base/lppl.txt; either
% version 1 of the License, or (at your option) any later version.
%
% This program includes other software that is licensed under the
% terms of the LPPL and the Perl Artistic License; see README for details.
%
% You, the user, still hold the copyright to any document you produce
% with this software (like your dissertation).


%% put lstnewenvironment declarations here, if you're using listings

%% end lstnewenvironment declarations

%% I put convenience definitions that will go in several chapters here

%%%%% begin convenience definitions

\makeatletter
\newcommand{\wb@episource}{}
\newenvironment{wbepi}[1]{\begin{quote}\renewcommand{\wb@episource}{#1}\itshape}{\par\upshape \raggedleft --- \textsc{\wb@episource}\\ \end{quote}}
\makeatother

%%%%% SVN
\IfFileExists{svn-multi.sty}{%
\usepackage{svn-multi}%
%%% Uncomment the second one and comment out the first one if you want
%%% to include subversion revision information in each file.
\newcommand{\vcinfo}{}%
%\newcommand{\vcinfo}{\begin{centering}\fbox{\fbox{\parbox{5in}{Author: \svnauthor\\Revision: \svnfilerev\\Last changed on: \svnfiledate\\URL: \svnkw{HeadURL}}}}\\[1em]\end{centering}}%
}{%
\newcommand{\svnidlong}[4]{}%
\newcommand{\svnfilerev}{}%
\newcommand{\svnauthor}{}%
\newcommand{\svnfiledate}{}%
\newcommand{\svnkw}{}%
\newcommand{\vcinfo}{}%
}

%TeXcount macros
\newcommand{\detailtexcount}[1]{%
  \immediate\write18{texcount -merge -sum -incbib -dir #1.tex > #1.wcdetail }%
  \verbatiminput{#1.wcdetail}%
}

\newcommand{\quickwordcount}[1]{%
  \immediate\write18{texcount -1 -sum -merge #1.tex > #1-words.sum }%
  \input{#1-words.sum} words%
}

%   -sum, -sum=   Make sum of all word and equation counts. May also use
%              -sum=#[,#] with up to 7 numbers to indicate how each of the
%              counts (text words, header words, caption words, #headers,
%              #floats, #inlined formulae, #displayed formulae) are summed.
%              The default sum (if only -sum is used) is the same as
%              -sum=1,1,1,0,0,1,1.


\newcommand{\quickcharcount}[1]{%
  \immediate\write18{texcount -1 -sum -merge -char #1.tex > #1-chars.sum }%
  \input{#1-chars.sum} characters (not including spaces)%
}

%%macros to reduce typing
\newcommand{\partialt}{\frac{\partial}{\partial t}}
\newcommand{\ecb}{$\vec{E}\times\vec{B}$\ }
\newcommand{\mecb}{\vec{E}\times\vec{B}}
\newcommand{\dal}{D_{\alpha}}
\newcommand{\chisq}{$\chi^2$}
\newcommand{\adhoc}{\textit{ad hoc\ }}


%%%%% end convenience definitions
